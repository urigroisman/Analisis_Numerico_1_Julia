\documentclass[11pt,a4paper]{article}

% ------------------------------------------------
% Idioma, codificación y fuentes
% ------------------------------------------------
\usepackage[spanish,es-nodecimaldot]{babel}
\usepackage[utf8]{inputenc}   % Entrada en UTF-8
\usepackage[T1]{fontenc}      % Acentos correctos en PDF
\usepackage{lmodern}          % Fuente Latin Modern
\usepackage{csquotes}         % Comillas tipográficas

% ------------------------------------------------
% Matemática
% ------------------------------------------------
\usepackage{amsmath,amssymb,amsthm,mathtools}

\numberwithin{equation}{section}  % Ecuaciones numeradas por sección

% Entornos tipo teorema
\theoremstyle{definition}
\newtheorem{ejercicio}{Ejercicio}
\newtheorem{definición}[ejercicio]{Definición}
\newtheorem{proposición}[ejercicio]{Proposición}
\newtheorem{ejemplo}[ejercicio]{Ejemplo}

% ------------------------------------------------
% Página y formato general
% ------------------------------------------------
\usepackage{geometry}
\geometry{margin=2.5cm}

\usepackage{setspace}
\onehalfspacing          % Interlineado 1.5

\usepackage{parskip}     % Espacio entre párrafos en lugar de sangría
\setlength{\parindent}{0pt}

% ------------------------------------------------
% Colores, gráficos y figuras
% ------------------------------------------------
\usepackage{xcolor}
\usepackage{graphicx}

% Paleta básica
\definecolor{linkcolor}{RGB}{0,0,120}
\definecolor{urlcolor}{RGB}{0,90,0}
\definecolor{codegray}{RGB}{40,40,40}
\definecolor{codebg}{RGB}{245,245,245}

% ------------------------------------------------
% Listas
% ------------------------------------------------
\usepackage{enumitem}
\setlist[itemize]{label=--,leftmargin=2em}
\setlist[enumerate]{leftmargin=2em}

% ------------------------------------------------
% Código fuente (Julia, etc.)
% ------------------------------------------------
\usepackage{listings}

\lstdefinelanguage{Julia}{
  morekeywords={
    abstract,break,case,catch,const,continue,do,else,elseif,end,exports,
    false,finally,for,function,global,if,import,let,local,macro,module,
    quote,return,true,try,using,while,mutable,struct
  },
  sensitive=true,
  morecomment=[l]\#,
  morestring=[b]"
}

\lstset{
  language=Julia,
  basicstyle=\ttfamily\small,
  keywordstyle=\bfseries\color{blue!60!black},
  commentstyle=\itshape\color{gray},
  stringstyle=\color{red!60!black},
  backgroundcolor=\color{codebg},
  frame=single,
  framerule=0.4pt,
  rulecolor=\color{black!40},
  numbers=left,
  numberstyle=\tiny,
  numbersep=5pt,
  tabsize=4,
  showstringspaces=false,
  breaklines=true,
  columns=fullflexible
}

% Comando cómodo para insertar fragmentos cortos de código en línea
\newcommand{\code}[1]{\texttt{#1}}

% ------------------------------------------------
% Hipervínculos
% ------------------------------------------------
\usepackage{hyperref}
\hypersetup{
  colorlinks=true,
  linkcolor=linkcolor,
  citecolor=linkcolor,
  urlcolor=urlcolor,
  pdfauthor={},
  pdftitle={Análisis de coste de \texttt{polyval}}
}

% ------------------------------------------------
% Título, autor y fecha
% ------------------------------------------------
\title{Análisis Numérico 1}
\title{Costo en multiplicaciones de la función \code{polyval}}
\author{Uri Groisman}
%\date{\today}

% ------------------------------------------------
% COMIENZO DEL DOCUMENTO
% ------------------------------------------------
\begin{document}

\maketitle

\begin{ejercicio}

\label{ej:costo-polyval}
  Supongamos que calcular $x^i$ cuesta $i-1$ multiplicaciones.
  Deducir el número total de multiplicaciones necesarias para evaluar,
  mediante el algoritmo \verb|polyval| definido arriba,
  un polinomio de grado $d$.
\end{ejercicio}


\bigskip     % espacio grande
\bigskip     % espacio grande
%
% ... solución
%
La función \code{polyval(c,x)} evalúa el polinomio en el punto $x$.
\begin{lstlisting}[language=Julia]
polyval(c, x) = sum([c[i]*x^(i-1) for i in 1:length(c)])
\end{lstlisting}
Supongamos que calcular $x^i$ cuesta $i-1$ multiplicaciones. Deducir
el número total de multiplicaciones necesarias para evaluar, mediante
este algoritmo, un polinomio de grado $d$.

\[
\texttt{length}(c)=d+1
\]
y que estamos evaluando
\[
p(x) = c_0 + c_1 x + c_2 x^2 + \dots + c_d x^d 
      = \sum_{k=0}^d c_k x^k,
\]
donde \(c_k = c[k+1]\).

El enunciado indica que calcular \(x^i\) cuesta \(i-1\) multiplicaciones. 
Queremos determinar cuántas multiplicaciones realiza el algoritmo \texttt{polyval} para evaluar
un polinomio de grado \(d\).

\medskip

\textbf{1. Costo de calcular las potencias de \(x\).}

Para cada término \(c_k x^k\) (con \(k=0,1,\dots,d\)) es necesario disponer de la potencia \(x^k\).
Consideramos el costo de calcular explícitamente cada potencia:

\begin{itemize}
  \item \(x^0 = 1\): no requiere multiplicaciones (\(0\)).
  \item Para \(k \ge 1\), por hipótesis, calcular \(x^k\) cuesta \(k-1\) multiplicaciones.
\end{itemize}

Por lo tanto, el número total de multiplicaciones para obtener todas las potencias
\(x^0,x^1,\dots,x^d\) es
\[
\sum_{k=1}^{d} (k-1).
\]

Renombrando el índice \(m = k-1\), cuando \(k\) recorre \(\{1,\dots,d\}\),
\(m\) recorre \(\{0,\dots,d-1\}\). Entonces
\[
\sum_{k=1}^{d} (k-1) = \sum_{m=0}^{d-1} m.
\]
Como se trata de la suma de una progresión aritmética,
\[
\sum_{m=0}^{d-1} m = \frac{(d-1)d}{2}.
\]

Así, el costo total en multiplicaciones asociado al cálculo de todas las potencias \(x^k\) es
\[
M_{\text{potencias}}(d) = \frac{(d-1)d}{2}.
\]

\medskip

\textbf{2. Costo de multiplicar por los coeficientes.}

Cada término del polinomio es de la forma
\[
c_k x^k.
\]
Una vez calculada la potencia \(x^k\), obtener el producto \(c_k x^k\) requiere una multiplicación.

Hay \(d+1\) coeficientes \(c_0, c_1, \dots, c_d\), por lo que se realizan
\[
M_{\text{coef}}(d) = d+1
\]
multiplicaciones para formar todos los productos \(c_k x^k\).

\medskip

\textbf{3. Número total de multiplicaciones.}

Sumando ambas contribuciones, el número total de multiplicaciones que realiza el algoritmo
\texttt{polyval} para evaluar un polinomio de grado \(d\) es
\[
N_{\text{mult}}(d)
= M_{\text{potencias}}(d) + M_{\text{coef}}(d)
= \frac{(d-1)d}{2} + (d+1).
\]

Desarrollamos y simplificamos:
\[
\frac{(d-1)d}{2} + (d+1)
= \frac{d^2 - d}{2} + \frac{2d + 2}{2}
= \frac{d^2 - d + 2d + 2}{2}
= \frac{d^2 + d + 2}{2}.
\]

\medskip

En conclusión, el número de multiplicaciones requerido por el algoritmo
\texttt{polyval} para evaluar un polinomio de grado \(d\) es
\[
\boxed{N_{\text{mult}}(d) = \dfrac{d^2 + d + 2}{2}.}
\]

Obsérvese que este costo crece como \(\mathcal{O}(d^2)\), en contraste con el método de Horner,
que evalúa el mismo polinomio usando únicamente \(d\) multiplicaciones y \(d\) sumas.

\end{document}