\documentclass[11pt,a4paper]{article}

% ------------------------------------------------
% Idioma, codificación y fuentes
% ------------------------------------------------
\usepackage[spanish,es-nodecimaldot]{babel}
\usepackage[utf8]{inputenc}   % Entrada en UTF-8
\usepackage[T1]{fontenc}      % Acentos correctos en PDF
\usepackage{lmodern}          % Fuente Latin Modern
\usepackage{csquotes}         % Comillas tipográficas

% ------------------------------------------------
% Matemática
% ------------------------------------------------
\usepackage{amsmath,amssymb,amsthm,mathtools}

\numberwithin{equation}{section}  % Ecuaciones numeradas por sección

% Entornos tipo teorema
\theoremstyle{definition}
\newtheorem{ejercicio}{Ejercicio}
\newtheorem{definición}[ejercicio]{Definición}
\newtheorem{proposición}[ejercicio]{Proposición}
\newtheorem{ejemplo}[ejercicio]{Ejemplo}

% ------------------------------------------------
% Página y formato general
% ------------------------------------------------
\usepackage{geometry}
\geometry{margin=2.5cm}

\usepackage{setspace}
\onehalfspacing          % Interlineado 1.5

\usepackage{parskip}     % Espacio entre párrafos en lugar de sangría
\setlength{\parindent}{0pt}

% ------------------------------------------------
% Colores, gráficos y figuras
% ------------------------------------------------
\usepackage{xcolor}
\usepackage{graphicx}

% Paleta básica
\definecolor{linkcolor}{RGB}{0,0,120}
\definecolor{urlcolor}{RGB}{0,90,0}
\definecolor{codegray}{RGB}{40,40,40}
\definecolor{codebg}{RGB}{245,245,245}

% ------------------------------------------------
% Listas
% ------------------------------------------------
\usepackage{enumitem}
\setlist[itemize]{label=--,leftmargin=2em}
\setlist[enumerate]{leftmargin=2em}

% ------------------------------------------------
% Código fuente (Julia, etc.)
% ------------------------------------------------
\usepackage{listings}

\lstdefinelanguage{Julia}{
  morekeywords={
    abstract,break,case,catch,const,continue,do,else,elseif,end,exports,
    false,finally,for,function,global,if,import,let,local,macro,module,
    quote,return,true,try,using,while,mutable,struct
  },
  sensitive=true,
  morecomment=[l]\#,
  morestring=[b]"
}

\lstset{
  language=Julia,
  basicstyle=\ttfamily\small,
  keywordstyle=\bfseries\color{blue!60!black},
  commentstyle=\itshape\color{gray},
  stringstyle=\color{red!60!black},
  backgroundcolor=\color{codebg},
  frame=single,
  framerule=0.4pt,
  rulecolor=\color{black!40},
  numbers=left,
  numberstyle=\tiny,
  numbersep=5pt,
  tabsize=4,
  showstringspaces=false,
  breaklines=true,
  columns=fullflexible
}

% Comando cómodo para insertar fragmentos cortos de código en línea
\newcommand{\code}[1]{\texttt{#1}}

% ------------------------------------------------
% Hipervínculos
% ------------------------------------------------
\usepackage{hyperref}
\hypersetup{
  colorlinks=true,
  linkcolor=linkcolor,
  citecolor=linkcolor,
  urlcolor=urlcolor,
  pdfauthor={},
  pdftitle={Análisis de coste de \texttt{polyval}}
}

% ------------------------------------------------
% Título, autor y fecha
% ------------------------------------------------
\title{Análisis Numérico 1}
\title{Demostración por inducción del algoritmo de Horner para evaluar polinomios}
\author{Uri Groisman}
%\date{\today}

% ------------------------------------------------
% COMIENZO DEL DOCUMENTO
% ------------------------------------------------
\begin{document}

\subsection*{Ejercicio 2}

\textbf{Enunciado.} Mostrar (por inducción) que la evaluación de un polinomio de grado
$d$ puede hacerse con $d$ multiplicaciones y $d$ sumas.

\medskip

Sea el polinomio
\[
p(x) \;=\; a_0 + a_1 x + a_2 x^2 + \dots + a_d x^d,
\qquad a_d \neq 0.
\]
Queremos probar que existe un algoritmo que, dados $x$ y los coeficientes
$a_0,\dots,a_d$, calcula $p(x)$ usando exactamente $d$ multiplicaciones y $d$ sumas.

\medskip

\textbf{Reescritura (forma de Horner).}
Todo polinomio de grado $d$ puede escribirse en forma anidada como
\begin{align*}
p(x)
&= a_0 + x(a_1 + x(a_2 + \dots + x(a_{d-1} + x a_d)\dots)) \\
&= (((a_d x + a_{d-1})x + a_{d-2})x + \dots )x + a_0.
\end{align*}
Esta forma sugiere el siguiente esquema iterativo:
\[
\begin{cases}
b_d = a_d,\\[2pt]
b_k = a_k + x\, b_{k+1}, & k = d-1, d-2, \dots, 0,
\end{cases}
\qquad\Rightarrow\qquad p(x) = b_0.
\]
En cada paso $b_k = a_k + x b_{k+1}$ se realiza exactamente
\emph{una} multiplicación ($x b_{k+1}$) y \emph{una} suma (sumar $a_k$).

\medskip

\textbf{Demostración por inducción.}

\textit{Proposición.} Para todo entero $d \ge 0$, la evaluación de un polinomio
de grado $d$ puede realizarse con $d$ multiplicaciones y $d$ sumas.

\medskip

\textit{Caso base $d=0$.}
Si $d = 0$, el polinomio es constante:
\[
p(x) = a_0.
\]
Evaluar $p(x)$ consiste simplemente en devolver $a_0$, sin realizar
multiplicaciones ni sumas. Esto coincide con la afirmación para $d=0$:
se requieren $0$ multiplicaciones y $0$ sumas. El caso base es válido.

\medskip

\textit{Hipótesis inductiva.}
Supongamos que la afirmación es verdadera para grado $d-1$:

\begin{quote}
Todo polinomio $q(x)$ de grado $d-1$ puede evaluarse usando exactamente
$d-1$ multiplicaciones y $d-1$ sumas.
\end{quote}

\medskip

\textit{Paso inductivo.}
Sea ahora $p(x)$ un polinomio de grado $d$. Podemos escribirlo como
\[
p(x) = a_0 + x\,q(x),
\]
donde
\[
q(x) = a_1 + a_2 x + \dots + a_d x^{d-1}
\]
es un polinomio de grado $d-1$.

\begin{itemize}
  \item Por la hipótesis inductiva, podemos evaluar $q(x)$ usando
        $d-1$ multiplicaciones y $d-1$ sumas.
  \item Una vez conocido $q(x)$, para obtener $p(x)$ realizamos:
        \begin{itemize}
          \item una multiplicación adicional: $x \cdot q(x)$,
          \item una suma adicional: $a_0 + x q(x)$.
        \end{itemize}
\end{itemize}

En total, para el polinomio $p(x)$ de grado $d$ usamos
\[
(d-1) + 1 = d \quad\text{multiplicaciones,}\qquad
(d-1) + 1 = d \quad\text{sumas.}
\]
Por lo tanto, la afirmación también es válida para grado $d$.

\medskip

Como el caso base $d=0$ es cierto y el paso inductivo
$d-1 \Rightarrow d$ es válido, por inducción matemática se concluye que:

\begin{quote}
Para todo $d \ge 0$, la evaluación de un polinomio de grado $d$ puede
realizarse con $d$ multiplicaciones y $d$ sumas.
\end{quote}

Este algoritmo corresponde precisamente a la evaluación en forma de Horner.

\end{document}